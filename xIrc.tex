\documentclass[titlepage]{article}
%\usepackage[dvips]{graphics}
\title{xIrc User Reference\\
Version 2.2}
\author{Joe Croft \\
joe@croftj.net \\
http://www.croftj.net/$\sim$xirc }
\begin{document}
\setcounter{tocdepth}{2}
\input{font}
\maketitle
\tableofcontents
\pagebreak[4]
\section{Introduction}
xIrc is, as the name implies, an IRC Client program that runs under X.
It is written in C++ and the binary for it provided was compiled using
gcc  2.7.2  and  qt-1.31  using  the  ELF  libraries.  See the INSTALL
document  provided  for  details  on building it if you need or desire
too, have no fear, it's reasonably easy to build.

I would appreciate any help
offered  in  finding any bugs that may crop up. This will be mostly in
the  form  of  turning on the various debug flags I have embedded then
mailing  me the output. Also, any suggestions for additions or changes
to  it's  look  and  feel will be appreciated. I cannot guarantee that
they will be implemented, but all will be taken into consideration.

Also, if you know of servers not in my .servers file or any corrections for
those that are, please mail me the changes for them and I will place them
in the file for future releases.

Note the email address change. The old address is still there but...

\subsection{Features}

\begin{itemize}
\item mIrc / CTCP2 Color support
\item CTCP2 Color and Text Attribute Support
\item Individual Windows for each channel you are joined to or each 
private conversation.
\item Nick Lists for the channels that allow you to various things on 
the folks listed in them.
\item Initiate and respond to DCC chat requests.
\item Initiate and accept DCC File transfers
\item Aliases that allow multiple words to be represented by a symbol.
Similar to the IRC-II set command.
\item Subset of similar commands and formats to IRC-II though many have
been simplified.
\item Auto responses to people getting kicked from a channel or for 
you being kicked from a channel.
\item Auto re-join if you've been kicked from a channel.
\item Simplified Banning w/ optional kick
\item Use of X-Resources for the configuration of fonts, colors and a
few other things.
\item Cut and Paste functionality at least for the input fields
few other things.
\item Member arrival notification either by nick or by address mask
\item Member ignore lists.
\end{itemize}

\subsection{Future Dreams and Wishes}
\begin{itemize}
\item Command line histories
\item Cut and Paste functionality for the chat windows in general.
\item Script language based on the one provided by IRC-II.
\item Having responses to CTCP Requests routed to the window in which they 
originated like the other commands such as \emph{WHO}, \emph{WHOIS}, etc.
\item Anything else that you or I may think of that sounds handy, 
nifty or just plain fun to have.
\end{itemize}

\subsection{Known Bugs}
In the notification and ignore tables, bot the address and nick fields 
must have something in them otherwise the program may crash.

Also in the notification and ignore tables, new entries are not recognized 
until the table is saved then reloaded.

\subsection{Credits}
Troll Tech, the writers of the 'qt' class library used to make the
X-Windows programming of this much easier.

The many folks who have given me ideas and inspirations for improving
xIrc.

While I am at it, I would like to  make  an  unsolicited  plug  for
the folks who wrote the GUI class library I used, qt. This library was
developed by the good folks at Troll Tech. Not only have they seen fit
to provide the class library to us Linux folks for free, but they have
also been invaluable in helping me making my program more
bullet-proof. The library has a nice Motif like look and feel to it
while it runs considerably lighter then Motif ever could and it
greatly simplifies the writing of XWindows applications. They can be
reached at info@troll.no and the URL to their home-page is
http://www.troll.no/.

Any of you programmers out there might want to check it out if you are
planning to do any X-Windows apps in the near future. Though it is not
free fro most platforms, nor for use in commercial pursuits, it is free
to the Linux community for non-commercial use. I have found it provides
a nice interface the greatly simplifies the creation of windows and
such.

\section{Installation and Compilation} \subsection{Installation}
 
xIrc is dynamically linked, you need to install the qt library from Troll 
Tech.  Please read their documentation closely so that you get the right 
version (currently 1.31)!  If you are not recompiling my program you need 
the latest version compiled for gcc-2.72 (This is the 'standard' release).  
Otherwise you need to select the version for the version of compiler you 
are using.  It sounds intimidating but if you read the docs supplied with 
qt carefully, it's not hard!  

Next, copy the file xIrc.ppm to where you want it to reside. xIrc will
look in the directories /usr/local/lib/xIrc, ./, and your home directory by
default for this file. If you choose another home for it, you will
have to place an entry, PixMapPath, in one of the resource files
telling xIrc where too look. See the section X-Resources.

Move and rename the the file servers.dat to /usr/local/lib/xIrc/.servers.
If you want to keep it elsewhere, The entry \emph{*Servers.Path} and
\emph{*Servers.Filename} will need to be changed in the resources file.

If you are using fvwm window manager, I have supplied xIrc.xpm for the
button's icon. I'll leave it to you to investigate setting up .fvwmrc
to put all this to use.

\subsection{Re-compilation}
Change to the xIrc-X.YY directory. Edit the
entries QTINCDIR, QTLIBDIR to point to where the Qt include files and Qt libraries reside
Then run \textit{make
depend} then \textit{make}. This too can be done on a single line as
\textit{make depend; make}. After this completes, follow the
installation notes for putting it and it's icon in their proper homes.

These files should compile w/ no errors or warnings. If they get
either, I don't know what to tell you. I would suspect an older
version of compiler or something. You can always email me and I'll
help you as I can.

\section{Usage}
xIrc does not have any command line switches of it's own. The QT library
may define a few of its own to change its look and feel, though I'm not sure
what all are available.

\subsection{Connecting to a Server}
Upon starting, a dialog asking to choose your nick will be displayed as
well as the main window. A default set of nicks are given. After selecting
a nick, which you must do, a table of servers will be presented. Select
your server of choice by double clicking the mouse on it. If your favorite
server is not in the list, you may add it by selecting the \emph{New} entry
under the \emph{Server} menu.

A small dialog will be displayed allowing you to change port numbers of
manually type in a different server name. Once you are done, click on the
\emph{Connect} button.

At this point, xIrc will attempt to connect to the selected server. After
it connects, any welcoming messages and the Message of the day will be
displayed on the screen.

\subsection{Entering a Channel}
To enter into a channel, select the \emph{Open Channel} entry in the
\emph{File} menu. You will be presented with a list of channels to enter and or
people to chat with. Either select a name from the list or type it in in
the edit field and press either \emph{OK} or \emph{Join} buttons. If you
want to see who a person is or who is on a channel, press the {Names/Whois}
button.

Once a channel is entered a new window will be displayed for that channel.
At this point any of the commands listed below may be entered or any
non-command will be sent to the channel or person.

If a channel was entered, a list of the members will be displayed for the
channel. If you find the list irritating as I do, you can press the
\emph{Nick List} button at the top of the channel window, it will toggle
the display of the member list for the channel.

\subsection{DCC Chat}
\subsubsection{Initiating a DCC Chat Connection}
A DCC chat is initiated by either opening the \emph{Channels \& People}
dialog by selecting the \emph{Open Channel} entry of the \emph{File} menu,
selecting the person or entering in their nick followed by pressing the
\emph{DCC Chat} button. Or by double clicking on the nick which will open
the \emph{Nick Action} dialog and pressing the \emph{DCC Chat} button. In
either case, a DCC Chat request will be sent to the nick, and a window will
be opened for the conversation. When the other end connects, a small
message will be displayed in the message window and you may be begin
chatting with them.

\subsubsection{Accepting a DCC Chat Request}
When a DCC Chat request comes in, a message box will be displayed
indicating who wants to chat with you. At this point you may either choose
to connect to them or open a window to chat privately with them. 

\subsection{Ignoring People}
Once in a while, jerks do enter in the scene and it is nice to be able to
ignore them. This may be done in one of two ways. Either double click on
their nick and when the \emph{Nick Action} dialog is presented, press
ignore. This will do a \emph{/who} command on them and build an appropriate
address filter for the person and present you with a dialog showing the
results. Press the \emph{OK} button to add the entry to the list.

The second means is to enter the name in manually. Select the \emph{Ignore
List} entry in the \emph{Lists} menu of the main window. This will display
the \emph{Ignore List}. You may at this point double click on an empty row
or select the \emph{New} entry under the \emph{Nick} menu of the table.
This will open an dialog allowing you to enter in the nick and or address mask
of the person.

Any message you enter in will be sent to the person each time xIrc receives
a message from them. This includes messages from them to a channel you are
in.

\subsection{Notifications}
The mechanics of setting up notifications of peoples comings and goings are
identical to that of the \emph{Ignore} lists. The only difference is the
use of the message field. It will be sent to the person whenever they first
detected to be online. Kind of an auto-greets message.

\subsubsection{By Nick or by Address}
Both ignores and notifications allow you to select whether they are by nick
or by address. Pressing the \emph{By Addr} button indicates that it should
be done by address. Otherwise, they are done by nick.



\section{Windows, Dialogs and Buttons}

There are several windows that may or may not be present at anytime
through a session. Th following is a brief description of most of
them.

\subsection{Main Window}
This window is primarily used for dealing with the server. If a
message window, for either a channel or an individual, cannot be found
for a message, it will be displayed in this window.
\subsubsection{Menus}
Enough actions have been added that I finally had to replace the buttons
with menus. 
\begin{description}
\item[File] Server and Channel related functions
\begin{description}
\item[Connect Server] Opens the Server Select Dialog (\ref{Server Select})
allowing you to connect to or disconnect from a server.
\item[Open Channel] Opens a dialog (\ref{Channel Select}) for joining a
channel a channel or starting up a chat with an individual.
\item[Quit] Opens up a dialog allowing you to select an exiting statement
then exits the program
\end{description}

\item[Nick] Nick related functions
\begin{description}
\item[Show Folks Online] Toggles the visibility of the \emph{Folks Online}
list (\ref{Folks Online}).
\item[Actions] Opens the 'Nick Action' dialog (\ref{Nick Action}). 
\item[Change Nick] Opens a dialog for changing your nick. If you select
this, you will be forced to enter in a nick.
\end{description}

\item[Lists] List Access / Maintenance
\begin{description}
\item[Server List] Same as the Connect Server menu item under the File
menu.(\ref{Server Select})
\item[Ignore List] Allow the editing and removing of existing, and the
entry of new members on the ignore list (\ref{Ignore List}). 
\item[Notify List] Allow the editing and removing of existing, and the
entry of new members on the notification list (\ref{Notify List}). 
\end{description}
\end{description}

\subsection{Message Window}
This Window is the window used for talking to individuals and
channels. There can be many such windows open at any given time. The
only real differences in the windows for chatting on a channel and
those for chatting to individuals is their initial size when they are
created. The windows for talking on a channel are created with about
twice as many lines.

Also found in this window is an edit field for entering commands and
messages to send. The commands and instructions on what can be entered
within this field will be explained in later sections.

To simplify actions on other's, double clicking on a nick in message 
window will open the \emph{Nick Actions} (\ref{Nick Action}) dialog box 
placing both the nick
and channel name, if any, into it for you. At that point, any 
number of actions are available to you.
 
\subsubsection{Buttons}
\begin{description}
\item[Close] Pressing this button parts (leaves) the the channel and
closes the window. In the case of a window for speaking to an
individual, the window is merely closed.
\item[Ping] Pressing this button send a CTCP Ping command to the
individual or channel.
\item[Nick Actions] Pressing this button opens the same dialog box as
the \textit{Nick Actions} button of the Main window.
\item[Nick List] This button is only present on the Message Windows
for channels. Pressing it either hides or shows a dialog box with the
names of the members currently on the channel.
\item[Clear Line] Pressing this button erases any text currently
present in the edit field.
\item[Show Colors] Pressing this button displays the current colors and the digit used
to change to that color.
\end{description}

Three Check Boxes have been added to this window. Checking them does as follows.

\begin{description}
\item[mIrc Colors] mIrc color coding is used when this box is checked. Otherwise the
forthcoming CTCP2 color coding is used.
\item[Ctcp2] When this button is checked, CTCP2 text attribute coding is used for underline,
inverse, bold etc. Otherwise, CTCP1 coding is used.
\item[Beep on Message] When this box is checked, any message to the window will cause it
to beep.
\end{description}

\subsection{DCC Chat Window}
This window is for chatting to another person via the CTCP DCC Chat
mechanism. It has the \textit{Close} and the \textit{Clear Line}
buttons. These buttons have the same functionality as the same buttons
on the \textit{Message Window}

\subsection{Server Selection Dialog}\label{Server Select}
This Dialog Box is for establishing and dissolving connections with the Irc
servers and manipulating the entries for the servers in the servers file.

The scrollbar, up/down arrow keys, page-up page-down keys, and the home \&
end keys can be used for traversing the list. Pressing the Enter key is
equivalent to pressing the \emph{Connect} button unless a change has been
made to the server selection masks.

The servers shown in the list can be limited by entering appropriate masks
into the editable fields at the top of the table. These masks are used in a
wildcard mode as in file selections. Pressing the Enter key after modifying
any of these fields will update the table to display The only those entries
that match the masks.

\subsubsection{Buttons}
This dialog has 3 buttons, \emph{Connect}, \emph{Disconnect}, and \emph{Cancel}. 

\begin{description}
\item[Connect] This pressing this button opens the Connection Dialog which
allows you to enter in a different server name or IP address and / or
select the port in which to connect to.
\item[Disconnect] Pressing this button disconnects the server.
\item[Cancel] Pressing this button closes the dialog box without any
action.
\end{description}

\subsubsection{Menu Bar}
The menu bar has two items, \emph{File} and \emph{Server}. The \emph{File}
menu has entries for manipulating the server file. The \emph{Server} menu
has entries for manipulating individual entries in the table.

These are the entries in the \emph{File} menu.
\begin{description}
\item[New] This menu entry clears the table of all entries
\item[Load] This menu entry opens a file dialog to select and load a file
of servers into the table.
\item[Import] This menu entry is like \emph{Load} except that it imports
server files in the format of mIrc's servers.ini files.
\item[Save As] This menu item saves the table to a file.
\item[Done] This menu item behaves as the \emph{Cancel} button.
\end{description}

These are the entries in the \emph{Server} menu.
\begin{description}
\item[Connect] This menu entry behaves as the \emph{Connect} button.
\item[Disconnect] This menu entry behaves as the \emph{Disconnect} button.
\item[Edit] This menu entry opens a \emph{Server Entry Edit} dialog box for
editing an entries contents.
\item[Delete] This menu item deletes an entry from the table.
\end{description}

\subsubsection{Server File format}
The file format for the servers file is quite simplistic by nature. It is a
colon delimited file with the following fields.

\begin{description}
\item[Group] Name of the group of servers the server belongs to.
\item[Country] Name of the country or continent where the server is located.
\item[State] Name of the state or country where the server is located.
\item[City] Name of the city where the server is located.
\item[Server] Domain name or IP number of the server.
\item[Ports] Comma delimited list of ports that can be connected to.
\end{description}

An example of a valid entry is
\begin{verbatim}
BeyondIRC:US:MI:St.Louis:irc.primary.net:6660,6662,6664,6666,6668
\end{verbatim}

\subsection{Server Connection Dialog}
This is a small dialog that gives the user an opportunity to change the
server name and select the port to connect to the server before connecting
to the server. This keeps one from being limited to only those servers and
ports that are specified in the server list.

There are two editable fields, one of which is a combo-box, the server
field and the port field.

There are also two buttons. The Connect button starts the connection, while
the Cancel button closes the window without any further action.

\subsection{Server Edit Dialog}
This dialog allows for updating information in the Server Table. 

\subsection{Nick List Dialog}
This dialog box is created and maintained for each \textit{Message
Window} opened for a channel. It holds the list of members of the
channel and a series of buttons for various actions. If a nick is not
present in the edit field, pressing a button will perform it's action
on the currently highlighted entry in the list. Each button's action
should be self evident by it's name but I'll explain them just in
case.

\subsubsection{Buttons}
\begin{description}
\item[Who is] Perform a \textit{/whois} command on the selected nick.
\item[Chat] Opens the \textit{Channels / People} dialog box so a private
chat or a DCC chat may be initiated with the selected member. The
member's nick will be put in the edit field of the \textit{Channels / People}
dialog box.
\item[Ping] Send a CTCP Ping command to the selected nick.
\item[Invite] Invite the selected nick (Uh, only makes sense if you
entered a non-members nick in the edit field) to the channel.
\item[Kick] Open the \textit{Kick Message} dialog box to kick a member
from the channel. This only works if you are a channel operator.
\item[Clear] Erases the contents of the edit field.
\item[Close] Should close the window but I don't think it does anything at
this point. *Sheepish Grin*
\end{description}

\subsection{Nick Actions Dialog}\label{Nick Action}
One of my favorite dialog boxes. This dialog box allows you to do all
sorts of things. On the buttons that require a channel to be entered,
you will receive an error if the channel field is blank.

Both the Nick and Channel fields will be filled in by double clicking
on a word in a \textit{Message Window} or the \textit{Main Window}.
This word selected will be placed in the nick field and the name of
the window will be placed in the channel field.

\subsubsection{Buttons}
\begin{description}
\item[Dcc Chat] Initiates a DCC Chat connection with the nick entered
in the nick field and opens a corresponding \textit{DCC Chat Window}.
\item[Priv Chat] Opens a \textit{Message Window} for the nick entered
in the nick field.
\item[File Send] Opens a file selection dialog for selecting the file to 
send. After the file is selected, it will be sent.
\item[User Info] Sends a CTCP UserInfo command to the nick entered.  
\item[Whois] Sends a \textit{/whois} command for the nick entered.
\item[WhoWas] Sends a \textit{/whowas} command for the nick entered.
\item[Finger] Sends a CTCP Finger command to the nick entered.
\item[Ban] This command sends a \textit{/who} command on the nick
entered, then builds an appropriate ban string and bans that person
from the channel specified.
\item[Kick] Same as the \textit{Ban} button except it kicks the person
from the channel once they are banned. You will be given a dialog box
to enter an appropriate message to go with the kick also. This button
is always good for livening up a slow channel.
\item[Time] Sends a CTCP Time command to the nick entered.
\item[Give Ops] Performs a \textit{/mode +o} command for the nick and
channel entered.
\item[Take Ops] Performs a \textit{/mode -o} command for the nick and
channel entered.
\item[Ping] Sends a CTCP Ping command to the nick entered.
\item[Who] Sends a \textit{/who} command for the nick or list of names
or whatever entered in the nick field. I found this to be the most
useful for seeing in any of my buds are online by entering a list
of domain name masks for each of them then pressing the button.
\item[Notify] Opens an edit dialog with a template \emph{Nick} and
\emph{Address Mask} for inclusion in the \emph{Notify} list. This command
may respond a bit slow because it performs a \emph{/who} command on the
nick to build its \emph{Address Mask}.
\item[Ignore] Opens an edit dialog with a template \emph{Nick} and
\emph{Address Mask} for inclusion in the \emph{Ignore} list. This command
may respond a bit slow because it performs a \emph{/who} command on the
nick to build its \emph{Address Mask}.
\item[Close] Closes the dialog box.
\end{description}

\subsection{Channels / People Dialog}\label{Channel Select}
This dialog box is for joining channels and initiating either private
or DCC chats with individuals. In Either case, an appropriate
\textit{Message Window} or \textit{DCC Chat Window} will be created.
As with the other dialog boxes with combo lists in them, pressing a
button will perform it's action on either the name in the edit field,
or the highlighted entry in the list.

\subsubsection{Buttons}
\begin{description}
\item[OK] This button either joins a channel or opens a
\textit{Message Window} to the selected individual.
\item[Join] Does the same as the \textit{OK} button.
\item[DCC Chat] Initiates a DCC Chat session with an individual and
opens a \textit{DCC Chat Window}.
\item[Chat] Does the same as the \textit{OK} button.
\item[Names/Whois] Sends either a \textit{/names} or \textit{/whois}
command to the selected channel or individual.
\item[Clear] Clears the edit field.
\item[Close] Closes the dialog box.
\end{description}

\subsection{Nickname Dialog}
This dialog lets you select or change your nick. As the others, it
will either use the nick entered in the edit field, or the selected
nick in the list. If the \textit{Auto Nick Selection} button is
enabled, the next nick found in the list will be used if the nick you
attempt to change to is in use. At this time, if you bring up this
dialog box, you \textbf{MUST} exit with the \textit{OK} button.

\subsection{Folks Online List}\label{Folks Online}
This is a small dialog showing the people from the \emph{Notify List}
(\ref{Notify List}) who are presently online. The nick that they are using
and the address as specified in the list are displayed for each person.

Double clicking the mouse on an entry in the list will pop up the
\emph{Nick Action} dialog box (\ref{Nick Action}) with the nick shown.

\subsection{Ignore / Notify List Maintenance Dialogs}
The list maintenance dialogs for both the Ignore and Notify Lists are quite
similar in their look and feel. Though a few names may differ, they a
basically the same dialogs. They consist of a series of menus and a
scrollable table with 4 columns.

\subsubsection{Ignore List}\label{Ignore List}
The ignore list is for ignoring incoming messages from individuals. When a
message from an individual on the ignore list is received, an optional
auto-reply message may be sent back if one is present in the \emph{Message}
field for that person. 

Currently, messages from ignored people are displayed in the main window.
This will most likely change as I become more confident that it is working
properly.

The only short coming of the auto-reply is that it will be sent to the
person when they chat on a channel you are currently on. Of course,
depending on your nature, this could be a good thing (snicker).

\subsubsection{Notify List}\label{Notify List}
The \emph{Notify List} is to receive notification of a persons arrival or
departure from Irc. Upon detecting a persons arrival, it will send an
optional greeting to that person and also post a message to the \emph{Main
Window} of their arrival. When a person has been detected as having
departed, another message will be posted to the \emph{Main Window} stating
so.

Detecting people is done with the \emph{who} command using either their
nick or the name of their provider. They both have potential problems. In
the case of using their nick, the problem is (at least to me) obvious,
anyone can use a given nick so a notification may or may not be true. In
the case of using their hostname the problem is not so obvious. Doing a
\emph{/who} on some hosts, there may be many people online from that host.
\emph{xIrc} gleans out the desired person by their username, but if there
are enough people on form a single hostname, you may get bumped because of
the number of responses from the server. Yes, this is a real problem. if
you want to see it happen, do a \emph{/who} with no name on a busy night.
The list will be so long, chances are the server will knock you offline.

\subsubsection{List Navigation and Manipulation}
An entry is selected when its row appears to be sunken in respect to the
other rows. Selecting a given row is a simple matter of using the up and
down arrows or the page up and page down keys to select the next or
previous item in the list. An item may also be clicked on with the mouse to
select it.

As well as using the menus described below to manipulate an item, the
mouse may also be used. Double clicking on an item will cause an edit
dialog to be opened for editing the contents. Double clicking on an
unpopulated row will open a dialog for entering a new item.

\subsubsection{Menus}
\begin{description}
\item{File}
\begin{description}
\item[New] Clears the list of all entries.
\item[Load] Reads in an existing file appending it's items in to the
list.
\item[Save As] Saves the list to a file. The default filename given is that
of the file that gets read on startup of the program.
\item[Done] Closes the dialog.
\end{description}

\item{Nick}
\begin{description}
\item[Edit] Opens a dialog for editing the currently selected item in the
list.
\item[New] Opens a dialog with blank entries for appending a new item to
the list.
\item[Delete] Deletes the currently selected item from the list.
\end{description}
\end{description}

\subsubsection{Fields}
Each item in the list consists of 4 fields. Only the 3rd field differs
between the notification and ignore lists and then only by it's label.

\begin{description}
\item[Nick] Nick of the user to be ignored or notified of.
\item[Mask] Address Mask of the user to be ignored or notified of. The
format of this field may be any string that makes for a valid /who query.
In general, it will have a format of \emph{nick}@\emph{server}. because
most people use dynamic IPs, the server will be prefixed with a *. As an
example:

\begin{quote}
peteh@*.satelnet.org
\end{quote}

\item[Ignore/Notify] Checkboxes to enable / disable the item and to select
if the user is to be detected by their nick or by their address mask.
\item[Message] For the Ignore List, if a message is present, it will be
sent to the user automatically in response to any message detected by them.
This include any messages they send to a channel you happen to be on also.
(Yes, I can be quite nasty, snicker).

For the Notify List the message present in this field will be sent to the
user on detection of their arrival. It will be sent only once on each
arrival of the user.

In either case, if this field of empty, no message will be sent.
\end{description}

%%
%% Commands Chapter
%%
\section{Commands}
Commands may be entered in the edit field of the \textit{Message
Windows}. They are distinguished from messages by the line starting
with a '/' character. All white space at the beginning of the line is
skipped, therefore, if you want to 'fake' a command, you must put a
period or some other non-white space character before the '/',
otherwise the command will be invoked! The following commands are
available (Maybe others if I forget to add them to this list as I
release the code).

\begin{description}
\item[ALIAS \textit{tag} \textit{string}]
   This  command  allows you to have simple macro replacements. Though
   they  are very basic, they can be useful. Basically \textit{string} will
   be  inserted  anywhere  \$\textit{tag} is  found  on  an entered
   line. For example:
\begin{quote}
\begin{verbatim}
/alias hugs *Big Hairy Hugs*
/me gives so and so $hugs tightly
\end{verbatim}
\end{quote}
   will result in an action of '*\textit{nick} gives so and so *Big Hairy Hug*
   tightly'.
\item[AWAY \textit{message}]
   Sets  you  away.  The \textit{message} will be given to anyone who private
   messages you.
\item[AWAY] 
   An away command with no message clears out the away message 
   from the server.
\item[FINGER \textit{nick}]
   Sends a CTCP FINGER request to the nick.
\item[INVITE \textit{nick}]
   This invites a person to the channel associated with the window it is
   typed  in  on. It is functional but useless on windows used to chat
   with others privately
\item[JOIN \textit{channel}]
   One  of  the  commands  you  probably ought to leave alone. Use the
   'Channel  /  People' Dialog instead. It will join you to
   \textit{channel},
   but  the  routing  of  messages  will not right and therefore won't
   work!
\item[KICK \textit{nick}]
   Kicks  a  \textit{nick} from  a  channel. Once again, it's functional but
   useless if used from a \textit{Message Window} not associated with a 
   channel.
\item[LINKS]
   Lists the servers currently connected to your server. Don't ask me
   how to interpret it, I don't know!
\item[LIST]
   Lists  channels topic and such. Works just like the LIST command in
   IRC-II.
\item[MAP]
   Similar to the \textbf{/LINKS} command except that it gives a more
   'graphical' look to how the servers are connected. Though I still
   have to pretend I know what it is telling me.
\item[ME]
   Sends a CTCP ACTION message. In DCC Chat windows it will append the
   action to the characters \texttt{<=}.
\item[MODE \textit{modes}]
   Sets  the  modes  for the channel the window is associated with to
   \textit{modes}. It
   works  just  like the MODE command in IRC-II except that the channel
   name is derived from the window's name.
\item[NAMES {[}\textit{channel}{]}]
   Lists  the  members  on  a  channel.  If  channel is not given, the
   channel name will be used. If I'm not mistaken, in the latter case,
   no  results will be display, instead, they will be reflected in the
   Nick Dialog for that channel.
\item[NOTICE {[}\textit{nick}{]} \textit{string}]
   Sends a NOTICE to the Nick or channel specified.
\item[NICK]
   Changes your nick
\item[PART]
   Like the JOIN command, it there, but it's best to leave it be!
\item[PING]
   Pings  the  Member or Channel given. If neither are given, the name
   pinged is derived from the window's name.
\item[PRIVMSG]
    Sends a private message to the channel or member specified.
\item[QUIT]
   Closes  the  connection to the current server sending the optional
   message if present.
\item[TOPIC]
   Set  or  displays  the  topic  for the channel associated with the
   window.
\item[WHO]
   Does a who on the nick provided. Works like IRC-II.
\item[WHOIS]
   Does a whois on the nick provided. Works like IRC-II.
\item[WHOWAS]
   Does a whowas on the nick provided. Works like IRC-II.
\item[VERSION]
   Sends  a  CTCP Version message to the channel or nick provided. If
   no name is given, the name of the window is used.
\end{description}

\section{Command Line Entry and Edit Fields}
Because both the \textit{Message Windows} and the edit fields of the
dialog boxes share a common widget for text entry, they have the same
abilities when it comes to aliases and special character handling.
Though it is not guaranteed that all of the special characters are
appropriate or will even be acceptable in all of the dialog boxes.

\subsection{Handling of Special Characters}
Special characters may be entered using \textit{escape sequences}.
These are with the character $\backslash$ followed by one or more
characters. There are 5 flavors of escape sequences.

\subsubsection{Escape Sequences, Text Attributes}
These escape sequences specify various text attributes and colors to send. They
are cumulative in that they remain on until explicitly turned off. All except the
$\backslash$c escape sequence are toggles in that the first time it is encountered, the
attribute is turned on while the second time it is turned off. In the case of the $\backslash$c
escape sequence, it has a specific parameter to turn off the colors.

Not all of these will work with the font selected for your window. Though they will still
be sent out and hopefully be properly displayed on the receiver's end.

\begin{description}
\item[$\backslash$c\textit{colorspec}] This sequence will set the foreground 
and background colors as specified. Both the foreground and background colors
must be present in the sequence! See \textit{Text Attributes \& Color Specifications} 
(Section \ref{textattr}).

\item[$\backslash$b] This sequence will toggle the bold text attribute.
\item[$\backslash$u] This sequence will toggle the underline text attribute. 
\item[$\backslash$v] This sequence will toggle the invert attribute
\item[$\backslash$s] This sequence will toggle the strikeoute attribute
\item[$\backslash$n] This sequence will remove any previously set attributes
including colors.
\end{description}

\subsubsection{Escape Sequences, Embedding Control characters}
These escape sequences allow you to embed control characters into the 
text stream.
\begin{description}
\item[$\backslash$\textit{number}] This will convert the decimal
number \textit{number} to a single character. Ei. $\backslash$27 will
send an ASCII \textit{Esc} character.
\item[$\backslash$x\textit{hex number}] This will convert the hexadecimal
number \textit{hex number} to a single character. Ei. $\backslash$1b will
send an ASCII \textit{Esc} character.
\item[$\backslash$0\textit{octal number}] This will convert the octal
number \textit{octal number} to a single character. Ei. $\backslash$033 will
send an ASCII \textit{Esc} character.
\end{description}

\subsection{Aliases}
An alias may be used by entering \$\texttt{<tag>}. Where tag is either
a name previously defined using the \textit{/ALIAS} command or the
name of an environment variable. Tag must begin with a letter, and may
consist of letters, numbers and the underscore after that. The regular
expression for aliases is \texttt{\$[a-zA-Z\_]+[a-zA-Z0-9\_]*}. If an
alias is found on the line but not defined it is replaced with
"" (no quotes).

At this point, it is not possible to to immediately follow a \$ with a
letter such as \$A without it being interpreted as an alias. The
upside to this is that you may put in \$456.34 without any special
characters to pass on the \$.

The names for alias are NOT case sensitive. \$fred is equivilent to
\$Fred and \$FRED.

\subsection{Command Lines}
A command line is considered any line entered in the edit field of
either a \textit{Message Window} or a \textit{DCC Chat Window} that
starts with a /\textit{command}. Any leading white space on the line
before the / will be skipped. If the command is not recognized it is
quietly ignored.

Any line that does not begin with a / is considered a message and is
sent on to either the channel or the individual you are chatting with.

In either case, you may imbed aliases and escape sequences as needed.

\section{Text Attributes \& Color Specifications} \label{textattr}
Currently Text Attributes are in a state of flux. There is the original
CTCP1 specification, a preliminary CTCP2 specifiaction and now a more 
finalized CTCP2 Specification. xIrc only handles the first and last of 
the three.

For colors there is the old mIrc way of encoding them (Blech!) and now
the newer CTCP2 way. xIrc will support incomming and outgoing of either
of the two formats.

As far as I can tell, most clients currently use the CTCP1 encoding of the text
attributes and the mIrc encoding of the colors. Therefore the default is to
use these two means of encoding the text attributes and colors.

In all cases, when sending commands, the format is the same. The means
of encoding the attributes are controlled by check boxes on the message windows.
For automated messages such as the Kick message, CTCP1 and mIrc color encodings 
are used and at this time, there is now way to change that.

\subsection{Color Specifications}

To specify colors, use a $\backslash$c immediately followed by two hex digits.
the first digit specifies the foreground color and the second specifies the background
color. Sadly the digits are diferent depending on whether you are using mIrc colors or
CTCP2 colors. All of the alpha-hex digits (A - F) may be in either upper or lower case.
The color code for the two specifications are as follows.

\subsubsection{mIrc Colors}
\begin{quote}
\begin{description}
\item[0] White
\item[1] Black
\item[2] Navy Blue
\item[3] Green
\item[4] Red
\item[5] Brown
\item[6] Purple (Magenta)
\item[7] Orange
\item[8] Yellow
\item[9] Lt. Green
\item[A] teal (dark cyan)
\item[B] Lt. Cyan
\item[C] Lt. Blue
\item[D] Pink (Lt. Magenta)
\item[E] Grey
\item[F] Lt. Grey
\item[X] Turn off colors
\end{description}
\end{quote}

In addition, a '.' may be used for the background color to specify that it should
remain unchanged from what it currently is. The foreground color must always be
specified as a hex digit 0 - F. 

Ex. Entering in "$\backslash$cd1Hello$\backslash$c0. there$\backslash$cx my friend" will produce
the string "Hello there my friend" with the word "Hello" in the colors Pink on Black,
the word "there" in the colors White on Black and my friend in the default colors.

\subsubsection{CTCP2 Colors}
For CTCP2, either a hex digit indicating the color or a \# folowed by six hex digits
(just like that used in the X-resources to specify a color)
may be used to indicate the foreground and background colors.

\begin{quote}
\begin{description}
\item[0] Black
\item[1] Blue
\item[2] Green
\item[3] Cyan
\item[4] Red
\item[5] Purple (Magenta)
\item[6] Brown
\item[7] Lt. Grey
\item[8] Gray
\item[9] Lt. Blue
\item[A] Lt. Green
\item[B] Lt. Cyan
\item[C] Lt. Red
\item[D] Pink (Lt. Magenta)
\item[E] Yellow
\item[F] White
\item[X] Turn off colors
\end{description}
\end{quote}

In addition, a '-' may be used to force either the foreground or background
color to the default color and a '.' may be used to specify that the foreground
or background color should remain the same as it was.

Ex. 1: Entering in "$\backslash$cd0Hello$\backslash$cf. there$\backslash$cx my friend" will produce
the string "Hello there my friend" with the word "Hello" in the colors Pink on Black,
the word "there" in the colors White on Black and my friend in the default colors.

Ex. 2: Entering in "$\backslash$c\#ff00ff\#000000Hello$\backslash$c\#ffffff- there$\backslash$cx my 
friend" will produce the string "Hello there my friend" with the word "Hello" in the colors Pink
on Black, the word "there" in the colors White on the default background color and my friend
in the default colors.

\section{Configurations Using the xIrc.defaults File}
Currently there are two means used in the configuration of xIrc, the
xIrc.defaults file and the X-Resources. I'm
not sure if this will not change in the future or not. For Y'all who
have been using xIrc for a while, you will find that somethings that
used to be in the xIrc.defaults file have been moved and made into
X-Resources. In the future, others or all may be moved also.

Any aliases that you want to pre-define, may be entered into this
file. It will be read once on startup and then politely ignored
through the remainder of the session.

At this point  it  MUST reside in the current directory where you are
when you start xIrc. There are just a handful of various things you
can set at this point, hopefully it will not get too much larger in
time.

If you take a look at the defaults file provided, you will get an idea
of  how  it's formated. definitions can be on multiple lines using the
continuation  character  as the last character on the line before they
new  line.  Any  handy  little  definitions you want to be able to use
while  chatting can also be placed in this file and they will be there
ready for you to use when you start up.

Here  is  the  list  of  the  currently used tags. Not all of them are
needed,  though  some  do make life much easier. These names may be in
any case to simplifiy reading them.

\begin{description}
\item[BAN\_MESSAGE]
   String to use for the kick if the kick is implemented with a ban.

\item[CHANNELS]
   List  of  channels  or  nicks you want to regularly check out, chat
   with or whatever.

\item[DCC\_SEND\_RESPONSE]
   Response  message  you  want to send when someone tries to set up a
   DCC File transfer to you.

\item[EMAIL\_ADDR]
   Entry  to  specifiy  the  user  name  and  host  name provided when
   connecting to the server. It must follow the <user>@<domain> syntax
   or  you  stand the chance that your connection will be rejected. If
   this entry is not provided, the user name from the /etc/passwd file
   will  be  used and a dummy domainname of dummy.hostname.org will be
   used.

\item[KICKMESSAGE]
   Default message to send when you kick someone.

\item[KICKED\_OTHER\_RESPONSE]
   Message to send when a kick of someone is detected.

\item[KICKED\_YOU\_RESPONSE]
   Message to send someone when they kick you.

\item[NICK]
   Now Obsolete tag. See NICKS.

\item[NICKS]
   Allows  you  to  define a set of possible nicks to when connecting.
   Each nick should be seperated with a space.

\item[QUITMESSAGE]
   Default message to append to the QUIT command.

\item[REALNAME]
   Will  be  sent when first connecting to a server as your real name.
   Can  be  anything you like though if you're one of those snots that
   don't  like  people  being  able  to  see  just who they are really
   talking  too  (Personal  oppinion there). If it is not defined, the
   user\_name field of the /etc/passwd file will be used instead.

\item[SERVERS]
   List of server names you frequent.

\item[TCPPORT]
   Default TCP port, probably should be left at 6667.

\item[USERINFO]
   Whatever  you  want to tell the world about yourself when they do a
   ctcp USERINFO command of you.
\end{description}

\subsection{Obsolete Tags}
These tags are no longer used as of Version 1.15 and their
functionality has been moved to the X-Resources file. 
\begin{description}
\item[FRAME\_FG\_COLOR\_NAME]
\item[FRAME\_FG\_COLOR\_RGB]
\item[FRAME\_BG\_COLOR\_NAME]
\item[FRAME\_BG\_COLOR\_RGB]
\item[WINDOW\_FG\_COLOR\_NAME]
\item[WINDOW\_FG\_COLOR\_RGB]
\item[WINDOW\_BG\_COLOR\_NAME]
\item[WINDOW\_BG\_COLOR\_RGB]
\end{description}

\section{X-Resources}
As of version 1.15, xIrc uses X-Resources to set the colors and fonts
to be used when drawing the widgets. Almost all of its widgets have
both a class name and a resource name. Currently the names are the
same except that the class names have the first letter of each word
capitalized. The names and definitions of the font and color resources
come directly from the definition of the Qt class library. Therefore,
for a better understanding of them, it wouldn't hurt to take a look
at Qt documentation for the classes QColorGroup and QFont.

Currently xIrc does not use the resource definitions defined for the
root window. They are derived by merging the following sources in the
order given.
\begin{itemize}
\item Hardcoded program defaults
\item /usr/lib/X11/app-defaults/xIrc $^*$
\item $\sim$/.Xdefaults
\end{itemize}
\flushleft{
Note:
}
\begin{quote}
$^*$ This file name will be the same as the the name that xIrc was
invoked as. In otherwords, if a link name \textbf{fred} points to the
actuall xIrc program, the resulting file name used will be
/usr/lib/x11/app-defaults/fred.
\end{quote}

The result of this is that if no other source is given, then the hard
coded valuee will be used, otherwise, and values given in the
.Xdefaults will override those in the app-defaults file which will
override the hardcoded values.

To simplify the documenting of the resources and the widgets, I will
refer to all of them by thier class name. If you insist on using the
actual names, you can convert them by making the class name all lowercase.
\flushleft{Note:}
\begin{quote}
For now, it is recomended that only class names be used in the
resource files. This is because in the near future the names for each
widget will probably be changing and I'm not sure how yet. Where as I
see very little change (though anything is possible) of the class
names.
\end{quote}

\subsection{Default Values}
The xIrc program has the following defaults built in. They will be in
force unless over ridden by definitions in one of the aforementioned
files.
\begin{quote}
\begin{verbatim}
XIRC*PixMap: xIrc.ppm,
XIRC*PixMapPath: ./;~/;/usr/local/lib/xIrc/;,
XIRC*Font.Family: Helvetica,
XIRC*Font.Size: 12,
XIRC*Font.Weight: normal,
XIRC*Menu.Font.Weight: Bold,
XIRC*Popup.Font.Weight: Bold,
XIRC*PushBtn.Font.Weight: Bold,
XIRC*PushBtn.Background: #afafaf,
XIRC*Background: 0xc3c3c3,
XIRC*Forground: black,
XIRC*BaseColor: light yellow,
XIRC*TextColor: black,
XIRC*AutoFocus: false,

XIRC*MessageBox.Background: #800000,
XIRC*MessageBox.Forground: Yellow,
XIRC*MessageBox.BaseColor: #800000,
XIRC*MessageBox.TextColor: Yellow,
XIRC*MessageBox.Font.Weight: Bold,
XIRC*MessageBox.Font.Size: 18,
   
XIRC*ListBox.BaseColor: #b8b8b8,

XIRC*MultiLine.Window.Background: #000090,
XIRC*MultiLine.Window.HighColor: white,
XIRC*MultiLine.Window.TextColor: grey72,

XIRC*MessageDialog.Columns: 60,
XIRC*MessageDialog.Lines: 7,
XIRC*MessageDialog.PushBtn.Background: #afafaf,
XIRC*MessageDialog*MultiLine.Window.Background: #bfbfbf,
XIRC*MessageDialog*MultiLine.Window.BaseColor: #9f9f9f,
XIRC*MessageDialog*MultiLine.Window.TextColor: Navy Blue,

XIRC*MultiLine.Window.Font.Family: Fixed,

XIRC*DccChat.Input.Font.Family: Fixed,
XIRC*DccChat.Columns: 80,
XIRC*DccChat.Lines: 11,
XIRC*DccChat.MircColors: TRUE,
XIRC*DccChat.CTCP2: False,
XIRC*DccChat.Beep: FALSE,

XIRC*MsgChat.Input.Font.Family: Fixed,
XIRC*MsgChat.Columns: 80,
XIRC*MsgChat.Lines.Channel: 22,
XIRC*MsgChat.Lines.Nick: 11,
XIRC*MsgChat.MircColors: TRUE,
XIRC*MsgChat.CTCP2: False,
XIRC*MsgChat.Beep: FALSE,

XIRC*ServerDialog.Servers.Background: #bfbfbf,
XIRC*ServerDialog.Font.Family: Fixed,
XIRC*ServerDialog.ImportFile: servers.ini,
XIRC*ServerDialog.ImportPath: ./,
XIRC*ServerDialog.ImportFilter: *.ini,
XIRC*ServerDialog.Filename: .servers,
XIRC*ServerDialog.Path: /usr/local/lib/xIrc,
XIRC*ServerDialog.Filter: .*,
\end{verbatim}
\end{quote}

I have tried to keep the defaults to a minimum and as generic as possible
allowing for the greatest flexiblilty. If they cause any problems, I'm
open for suggestions on how they should be set, though I don't make
any promises everybody's suggestions wil be put into place.

\subsection{Common Resources}
Each widget has these resources named.
\begin{description}
\item[Background] Used to se the backgound color of the widget. On
many of the widgets, such as lists, this also includes any scroll bars.
\item[Foreground] Not sure what all the covers. The one thing that I
have noticed that it changes is the caret color in the editible
fields.
\item[BaseColor] Background color of the input fields. The Background
color will be used if this is not set.
\item[TextColor] Color of the text that is displayed in a widget. The
Foreground color will be used if this is not defined.
\item[Font.Family] Family name of the font to use, such as, Helvetica,
Courier, Fixed, etc.
\item[Font.Size] Pixel size of the font to use.
\item[Font.Weight] Weight of font to use. This can be either a number
from 1 to 99 or one of the following names.
\begin{quote}
\begin{itemize}
\item{Light}
\item{Normal}
\item{Bold}
\item{Black}
\end{itemize}
\end{quote}
\end{description}

\subsection{Configurable Widgets}

\begin{description}

\item[XIRC] The application. This has following resource names
in addition to the common ones.

\begin{description}
\item[BanDialog] The dialog box used for banning folks from a channel.
It has the \textbf{Input} and \textbf{PushBtnFrame} widgets beneath
it.

\item[ChannelDialog] The dialog box for selecting the channel to talk
on. It also has the \textbf{ComboBox} and \textbf{PushBtnFrame}
widgets beneath it.

\item[ChannelNickDialog] This dialog box holds the list of current
members of a channel and has buttons for various actions for the
nicks. It has the \textbf{ComboBox} and \textbf{PushBtn} widgets
beneath it.

\item[ComboBox] This is for widgets composed of a list box and an
edit field. It has the \textbf{Input} widget beneath it.

\item[DccChat] Window used for DCC Chats. This has the same
widgets beneath it as the \textbf{MsgChat} widget.

It also has the following resources that can be set.

\begin{description}
\item[Columns] This specifies the number of columns wide the window should be.
\item[Lines.Channel] This specifies the of lines that a window for a channel
should have.
\item[Lines.Nick] This specifies the the number of lines a window for private
chats should have
\end{description}


\item[DCC.Open] Default Directory to open for Dcc File Sends

\item[DCC.Save] Default Directory to open for Dcc File Receives

\item[ErrorDialog] The dialog box that pops up when an error is
recieved from the server.

\item[IgnoreDialog] The Dialog Box for viewing and altering the \emph{Ignore
List}. It has the following resources in addition to the standard
resources.
\begin{description}
\item[File] Default name for loading a file into the list.
\item[Filename] Name of the file to load on startup.
\item[Filter] Filter for the files to see in the file dialog.
\end{description}

\item[Input] This widget is a single line editable field.

\item[ListBox] Used to set the colors and fonts of the various list boxes.

\item[Main] Main text window that the server messages go to.

\item[MessageBox] The dialog for handling errors and warnings.

\item[MessageDialog] The dialog box that pops up when someone sends a
private message to you. It has the \textbf{PushBtnFrame} widget	and \textbf{MultiLine}
widget beneath it.

\item[MsgChat] The window you talk with to both a channel or another
person privately. Currently ther is no differentiation between the
two, though later there will probably be so. I will keep the ends of
the new class names \textbf{MsgChat} to simplify the modifications to
the resources when that time comes.

It has the following resources that can be set.

\begin{description}
\item[Columns] This specifies the number of columns wide the window should be.
\item[Lines.Channel] This specifies the of lines that a window for a channel
should have.
\item[Lines.Nick] This specifies the the number of lines a window for private
chats should have
\end{description}

The \textit{MsgChat} window also has the \textbf{MultiLine},
\textbf{PushBtnFrame}, \textbf{ChanNickDialog}, and \textbf{Input}
widgets beneath it.

\item[MultiLine.Frame] This specifies the color of the frame and
scroll bar sourounding the multiline text widget used for displaying
the text of the conversations. At this time, because of how the frame
is layed out, it really only effects the color of the scrollbar.

\item[MultiLine.Window] The multiline text widget used to display the
text of conversations. For the best results, a fixed font such as
lucidatypewriter or fixed should be used for this window. The code for
the widget was written on the assumption that this would be so,
therefore, all bets are off that the widget will display it's contents
correctly if a proportional font is used!

\item[NickDialog] The dialog box for selecting a nickname. It too has
both the \textbf{ComboBox} and \textbf{PushBtnFrame} widgets beneath it.

\item[NickActionDialog] My favorite, the dialog box to do just about
anything to anyone. It have the \textbf{Input} and \textbf{PushBtnFrame} widgets beneath
it.

\item[NotifyDialog] The Dialog Box for viewing and altering the \emph{Notify
List}. It has the following resources in addition to the standard
resources.
\begin{description}
\item[Menu] Used to set the colors and fonts of the menu bar.
\item[Popup] Used to set the colors and fonts of the popup menus.
\item[File] Default name for loading a file into the list.
\item[Filename] Name of the file to load on startup.
\item[Filter] Filter for the files to see in the file dialog.
\end{description}

\item[PixMap] Name of the Pixmap file to use as an Icon for closed
\textit{Message Windows} and \textit{DCC Chat Windows}. Unfortunately,
This file cannot be of the standard xpm variety. It must be a BMP,
XBM, or a PNM of the type P1 - P6. I have supplied both an XPM and a
P3 format PPM file derived from the XPM. The XPM version is for all of
you fvwm users. 

\item[PixMapPath] A list of paths, delimited with semicolons, to
search for the pixmap. The default is './;$\sim$/;/usr/local/lib'.
\end{description}

\item[PushBtnFrame] This frame is for holding one or more push
buttons. It has the \textbf{PushBtn} widget beneath it.

\item[ServerDialog] The dialog box for selecting a server to connect
to. It has the \textbf{ComboBox}, \textbf{PushBtnFrame} and
\textbf{Input} (for the port number) widgets beneath it. In addition to the
common resources it also uses these additional resources.
\begin{description}
\item[Menu] Used to set the colors and fonts of the menu bar.
\item[Popup] Used to set the colors and fonts of the popup menus.
\item[ImportFile] Default name for the mIrc servers.ini file
\item[ImportPath] Default path to find the mIrc file.
\item[ImportFilter] Default filter for the mIrc file.
\item[File] Default name for the server file
\item[Path] Default path to find the server file.
\item[Filter] Default filter for the server file.
\end{description}

\item[Servers] The Server Selection Table.
\begin{description}
\item[Mask.Group] Mask to restrict servers the shown by group
\item[Mask.Country] Mask to restrict servers the shown by country
\item[Mask.State] Mask to restrict servers the shown by state
\item[Mask.City] Mask to restrict servers the shown by city
\item[Mask.Server] Mask to restrict servers the shown by server
\end{description} Mask to restrict servers the shown by group

\item[SocketDialog] The dialog box showing the progress of the
connection attempt. It has the \textbf{PushBtnFrame} widgets beneath it.

\item[TCPSocket] The socket widgit for establishing and accepting
connections. This is truely not a widgit as it is never displayed on
the screen. This object has the following two configurable items.
\begin{description}
\item[ConnectTime] Time to wait to make a connection in milliseconds
(seconds * 1000). The defualt is 1 minute.
\item[AcceptTime] Time to wait for an incomming connection in
milliseconds (seconds * 1000). The default is 5 minutes.
\end{description}

\end{description}
\end{document}
